% !TEX TS-program = xelatex
% !TEX encoding = UTF-8 Unicode
% !Mode:: "TeX:UTF-8"

\documentclass{resume}
\usepackage{zh_CN-Adobefonts_external} % Simplified Chinese Support using external fonts (./fonts/zh_CN-Adobe/)
% \usepackage{NotoSansSC_external}
% \usepackage{NotoSerifCJKsc_external}
% \usepackage{zh_CN-Adobefonts_internal} % Simplified Chinese Support using system fonts
\usepackage{linespacing_fix} % disable extra space before next section
\usepackage{cite}

\begin{document}
\pagenumbering{gobble} % suppress displaying page number

\name{IT工匠}

\basicInfo{
  \email{xuefangang97@gmail.com} \textperiodcentered\ 
  \phone{(+86) 156-5180-8915} \textperiodcentered\ 
  \linkedin[fanggangxue]{https://www.linkedin.com/in/fanggangxue}}
 
\section{\faGraduationCap\  教育背景}
\datedsubsection{\textbf{南京食堂不错大学}, 南京}{2016 -- 至今}
\textit{在读学士}\ 软件工程, 预计 2020 年 6 月毕业


\section{\faUsers\ 实习/项目经历}


\datedsubsection{\textbf{项目1}}{2017年12月 -- 2018年12月}
\role{DeepLearning,Python,Android,Linux}{科创项目,和团队成员合作开发}
\role{项目负责人}{指导老师: xxx教授}
\begin{onehalfspacing}
基于xxx在手机上实现xxx,项目地址:https://github.com/DmrfCoder 

\begin{itemize}
  \item xxx,模型识别准确率95 \%
  \item xxx,识别率80\%
  \item \textbf{重点是要写清楚你的项目有哪些具体的数据优势}
\end{itemize}
\end{onehalfspacing}


\section{\faCogs\ IT 技能}
% increase linespacing [parsep=0.5ex]
\begin{itemize}[parsep=0.5ex]
  \item 编程语言: C == Java == Python > Dart > Kotlin
  \item 平台: Linux
  \item 开发: Android,Flutter,DeepLearning
\end{itemize}

\section{\faHeartO\ 获奖情况}

\datedline{\textit{参与奖}, xxx比赛}{2017 年12 月}


\section{\faInfo\ 其他}
% increase linespacing [parsep=0.5ex]
\begin{itemize}[parsep=0.5ex]
  \item 个人主页: https://ai-exception.blog.csdn.net
  \item 主要比赛实践经历:http://ai-exception.com/news/
  \item GitHub: https://github.com/DmrfCoder
  
  \item 语言技能:
\end{itemize}

%% Reference
%\newpage
%\bibliographystyle{IEEETran}
%\bibliography{mycite}
\end{document}